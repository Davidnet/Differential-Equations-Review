\documentclass[10pt,a4paper]{article}
\usepackage[utf8]{inputenc}
\usepackage{amsmath}
\usepackage{amsfonts}
\usepackage{amssymb}
\newtheorem{define}{Definition}
\newtheorem{thm}{Theorem}
\newthorem{methods}{Method of Solution}
\author{David Cardozo}
\title{Review of Differential Equations}
\begin{document}
\maketitle
These notes contains the main methods for solutions of differential equations which were covered in  the course \textit{Ecuaciones Diferenciales} at Universidad de los Andes.
\section{First Order Differential Equations}
\begin{define}
A first order differential equation of the form:
\begin{equation}
\dfrac{dy}{dx} = g(x)h(y)
\end{equation}
is said to be separable or to have separable variables.
\end{define}
\begin{thm}
The solution for a differential equation that is separable can be found by the following method:
\begin{align*}
\dfrac{dy}{dx} = g(x)h(y) \\
\dfrac{dy}{h(y)} = g(x)dx
\end{align*}
So that the general solution can be found as:
\begin{equation}
\int \dfrac{dy}{h(y)} = \int g(x) dx
\end{equation}
\end{thm}

\begin{define}
A first order differential equations of the form:
\begin{equation} \label{eq:linear}
a_1(x)\dfrac{dy}{dx} + a_0(x)y = g(y)
\end{equation}
and $a_1 \neq 0$ is said to be a linear equation in the variable y.
\end{define}
\textbf{Remark} Observe that the linear equation \eqref{eq:linear}
can be transformed into a more known form of the linear equation:
\begin{equation} \label{LinearDif}
\dfrac{dy}{dx} + P(x)y 	= f(x)
\end{equation}
by dividing \eqref{eq:linear} by $a_1(x)$ since $a_1(x) \neq 0$
\begin{thm}
A linear differential equation of the form:
\begin{equation*}
\dfrac{dy}{dx} + P(x)y 	= f(x)
\end{equation*}
has a solution of the form:
\begin{equation} \label{eq:linearsolution}
y(x) = \dfrac{\int \mu(x)f(x)dx}{ \mu(x)}
\end{equation}
where $\mu(x)$ is defined as:
\begin{equation*}
\mu(x) = e^{ \int P(x) dx}
\end{equation*}
\end{thm}
\textbf{Remark} In conclusion, observe that for solving a linear first order differential equations we take the following steps:
\begin{itemize}
\item Put linear equation into the form \eqref{LinearDif}.
\item Identify correctly $P(x)$ and find the integrating factor $ \mu(x) = e^{ \int P(x) dx}$. For easiness choose the integration constant to be zero.
\item Use \eqref{eq:linearsolution} to find the solution.
\end{itemize} 
So before we define of what it is a exact differential equation, we need the definition of an exact differential from vector calculus.
\begin{define}
A differential expression $ M(x,y)dx + N(x,y)dy $ is an \textbf{exact differential} if it corresponds to the differential of some function $f(x,y)$, that is, given a function of two variables, say $ z = f(x,y) $, it happens that $ dz =  \dfrac{\partial f}{\partial x}dx + \dfrac{\partial f}{\partial y}dy = M(x,y)dx + N(x,y)dy $.
\end{define}
\textbf{Remark}
Observe that in the special case that we take a function of the form $ f(x,y) = c $ where $c$ is any real number, the differential of $f(x,y)$ is:
\begin{equation*}
\dfrac{\partial f}{\partial x} dx + \dfrac{\partial f}{\partial y} dy = 0
\end{equation*}
Remember that the derivative of a number is zero. 

\textbf{Example}
Let $ f(x,y) = 5x^3 + 12xy + 5y^2 = c $, where $c$ is any real number, the differential of $f$ is $ (15x^2 + 12y) dx + ( 12x + 10y) = 0 $. Since $ \dfrac{\partial f}{\partial x} = 15x^2 + 12y $, $\dfrac{\partial f}{\partial y} = 12x + 10y$, and $ df = 0 $ since $ f(x,y) = c$ and the derivative of a number (in this case c) is zero. We say $ (15x^2 + 12y) dx + ( 12x + 10y)  $ is an exact differential, since is the differential of $ f(x,y) = 5x^3 + 12xy + 5y^2 = c$
\begin{define}
A first order differential equation of the form
\begin{equation} \label{eq:exact}
M(x,y)dx + N(x,y)dy = 0
\end{equation}
is said to be an \textbf{exact equation} if the expression on the left-hand side is an exact differential.
\end{define}
So the question at this moment is: given an expression of the form $M(x,y)dx + N(x,y)dy$. How we know is an exact differential? The following theorem provides an answer.
\begin{thm}
Let $M(x,y)$ and $N(x,y)$ be continuous and have continuous first partial derivatives. Then a necessary and sufficient condition that $M(x,y)dx + N(x,y)dy$ be an exact differential is:
\begin{equation}
\dfrac{\partial M}{\partial y} = \dfrac{\partial N}{\partial x}
\end{equation} 
\end{thm}
So that to check a first order differential equation is an exact equation, put the differential equation into the form \eqref{eq:exact} and check if the partial derivatives are equal.

Now if the differential equation is an exact equation the solution is found with the following method



\end{document}
