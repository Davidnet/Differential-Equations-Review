\documentclass[10pt,a4paper]{article}
\usepackage[utf8]{inputenc}
\usepackage{amsmath}
\usepackage{amsfonts}
\usepackage{amssymb}
\newtheorem{define}{Definition}
\newtheorem{thm}{Theorem}
\author{David Cardozo}
%German David Cardozo Bolivar, Germán David Cardozo Bolívar.
\title{Review of Differential Equations}

\newcommand{\LL}{\mathcal{L}}
\newcommand{\braces}[1]{ \left\lbrace #1 \right\rbrace}
\begin{document}
\maketitle
These notes contains the main methods for solutions of differential equations which were covered in  the course \textit{Ecuaciones Diferenciales} at Universidad de los Andes.
\section{First Order Differential Equations}
\begin{define}
A first order differential equation of the form:
\begin{equation}
\dfrac{dy}{dx} = g(x)h(y)
\end{equation}
is said to be separable or to have separable variables.
\end{define}
\begin{thm}
The solution for a differential equation that is separable can be found by the following method:
\begin{align*}
\dfrac{dy}{dx} = g(x)h(y) \\
\dfrac{dy}{h(y)} = g(x)dx
\end{align*}
So that the general solution can be found as:
\begin{equation}
\int \dfrac{dy}{h(y)} = \int g(x) dx
\end{equation}
\end{thm}

\begin{define}
A first order differential equation of the form:
\begin{equation} \label{eq:linear}
a_1(x)\dfrac{dy}{dx} + a_0(x)y = g(y)
\end{equation}
and $a_1 \neq 0$ is said to be a linear equation in the variable y.
\end{define}
\textbf{Remark} Observe that the linear equation \eqref{eq:linear}
can be transformed into a more known form of the linear equation:
\begin{equation} \label{LinearDif}
\dfrac{dy}{dx} + P(x)y 	= f(x)
\end{equation}
by dividing \eqref{eq:linear} by $a_1(x)$ since $a_1(x) \neq 0$
\begin{thm}
A linear differential equation of the form:
\begin{equation*}
\dfrac{dy}{dx} + P(x)y 	= f(x)
\end{equation*}
has a solution of the form:
\begin{equation} \label{eq:linearsolution}
y(x) = \dfrac{\int \mu(x)f(x)dx}{ \mu(x)}
\end{equation}
where $\mu(x)$ is defined as:
\begin{equation*}
\mu(x) = e^{ \int P(x) dx}
\end{equation*}
\end{thm}
\textbf{Remark} In conclusion, observe that for solving a linear first order differential equations we take the following steps:
\begin{itemize}
\item Put linear equation into the form \eqref{LinearDif}.
\item Identify correctly $P(x)$ and find the integrating factor $ \mu(x) = e^{ \int P(x) dx}$. For easiness choose the integration constant to be zero.
\item Use \eqref{eq:linearsolution} to find the solution.
\end{itemize} 
So before we define of what it is a exact differential equation, we need the definition of an exact differential from vector calculus.
\begin{define}
A differential expression $ M(x,y)dx + N(x,y)dy $ is an \textbf{exact differential} if it corresponds to the differential of some function $f(x,y)$, that is, given a function of two variables, say $ z = f(x,y) $, it happens that $ dz =  \dfrac{\partial f}{\partial x}dx + \dfrac{\partial f}{\partial y}dy = M(x,y)dx + N(x,y)dy $.
\end{define}
\textbf{Remark}
Observe that in the special case that we take a function of the form $ f(x,y) = c $ where $c$ is any real number, the differential of $f(x,y)$ is:
\begin{equation*}
\dfrac{\partial f}{\partial x} dx + \dfrac{\partial f}{\partial y} dy = 0
\end{equation*}
Remember that the derivative of a number is zero. 

\textbf{Example}
Let $ f(x,y) = 5x^3 + 12xy + 5y^2 = c $, where $c$ is any real number, the differential of $f$ is $ (15x^2 + 12y) dx + ( 12x + 10y) = 0 $. Since $ \dfrac{\partial f}{\partial x} = 15x^2 + 12y $, $\dfrac{\partial f}{\partial y} = 12x + 10y$, and $ df = 0 $ since $ f(x,y) = c$ and the derivative of a number (in this case c) is zero. We say $ (15x^2 + 12y) dx + ( 12x + 10y)  $ is an exact differential, since is the differential of $ f(x,y) = 5x^3 + 12xy + 5y^2 = c$
\begin{define}
A first order differential equation of the form
\begin{equation} \label{eq:exact}
M(x,y)dx + N(x,y)dy = 0
\end{equation}
is said to be an \textbf{exact equation} if the expression on the left-hand side is an exact differential.
\end{define}
So the question at this moment is: given an expression of the form $M(x,y)dx + N(x,y)dy$. How we know is an exact differential? The following theorem provides an answer.
\begin{thm}
Let $M(x,y)$ and $N(x,y)$ be continuous and have continuous first partial derivatives. Then a necessary and sufficient condition that $M(x,y)dx + N(x,y)dy$ be an exact differential is:
\begin{equation} \label{eq:criterionexact}
\dfrac{\partial M}{\partial y} = \dfrac{\partial N}{\partial x}
\end{equation} 
\end{thm}
So that to check a first order differential equation is an exact equation, put the differential equation into the form \eqref{eq:exact} and check if the partial derivatives are equal.

Now if the differential equation is an exact equation the solution is found with the following method:

\textbf{Method of solution for an exact equation}
Given an equation in the form:
\begin{equation*}
M(x,y)dx + N(x,y)dy = 0
\end{equation*}
First determine if the expression on the left hand side is an exact differential, \textit{i.e} the left hand side complies with the equality on \eqref{eq:criterionexact}. 

If it does, Observe that there exist a function $f$ such that:
\begin{equation*}
\dfrac{\partial f}{\partial x} = M(x,y)
\end{equation*}
So that, we can find $f$ by integrating $M(x,y)$ with respect to $x$ while holding $y$ constant:
\begin{equation*}
f(x,y) = \int M(x,y) dx + g(y)
\end{equation*}
Where $g(y)$ appears since it is the constant of integration (Observe that since we are integrating with respect to $x$ the constant of integration can be any function that depends only on $y$).

Now, if we differentiate the previous equation with respect to y, and assume that $ \dfrac{\partial f}{\partial y} = N(x,y) $:
\begin{align*}
\dfrac{\partial f}{\partial y} = \dfrac{\partial}{\partial y} \int M(x,y)dx + g'(y) = N(x,y)
\end{align*}
Isolating $g'(y)$ and integrating we find $g(y)$
\begin{equation} \label{eq:usingcriterion}
g'(y) = N(x,y) - \dfrac{\partial}{\partial y }\int M(x,y)dx \\
\end{equation}
So that:
\begin{equation*}
g(y) = \int (N(x,y) - \dfrac{\partial}{\partial y }\int M(x,y)dx)dy
\end{equation*}
And finally, we have the solution  of the equation given by $f(x,y)=c$.

Before going to do some examples, let us remark two things in this method of solution.
First, observe that the expression on \eqref{eq:usingcriterion} is independent of $x$, that because
\begin{align*}
\dfrac{\partial}{\partial x} \left[ N(x,y) - \dfrac{\partial}{\partial y} \int M(x,y) dx \right] = \dfrac{\partial N}{\partial x} - \dfrac{\partial}{\partial y} \left(  \dfrac{\partial}{\partial x} \int M(x,y) dx \right) = \dfrac{\partial N}{\partial x} - \dfrac{\partial M}{\partial y} = 0
\end{align*} 
So observe that the hypothesis that the differential expression is an exact differential is been used (that is why is so important to check for the exact differential criterion, that is, the equality \eqref{eq:criterionexact} holds).
Second, observe that we could have also started assuming that $ \dfrac{\partial f}{\partial y } = N(x,y) $. After integrating N with respect to $y$ and then differentiating that result, we would have found:
\begin{equation*}
f(x,y) = \int N(x,y)dy + h(x) \quad \text{ and } \quad  h'(x) = M(x,y) - \dfrac{\partial}{\partial x} \int N(x,y)dy
\end{equation*}
which are analogous to the equations we discussed in the method of solution for exact equations; usually you usd the last two relations when if integrating with respect to $x$ is rather difficult or impossible, whereas integrating to $y$ is much more easy.

It may seem to the reader that the previous discussion is rather complex and complicated, but once  she sees the examples, he would find exact equations easy:

\textbf{Examples}
Insert examples here

\begin{define}
The differential equation
\begin{equation}\label{eq:Bernoulli}
\dfrac{dy}{dx} + P(x)y = f(x)y^n
\end{equation}
where $n$ is any real number, is called \textbf{Bernoulli's equation.}
\end{define}
Observe that for the values $n=0$  or $n=1$, \eqref{eq:Bernoulli} is a linear equation for which we know how to solve. The following method of solution is used for when $ n \neq 0,1$.
\textbf{Method of Solution for Bernoulli Equation}

Assume we have the following differential equation:
\begin{equation*}
\dfrac{dy}{dx} + P(x)y = f(x)y^n
\end{equation*}

Since $y^n \neq 0$ for $n \neq 0,1$ we will divide $y^{n}$ and rewrite as:

\begin{equation} \label{eq:Bernoulli Substituion}
y^{-n}\dfrac{dy}{dx} + P(x)y^{1-n} = f(x)
\end{equation}
Observe $ \dfrac{y}{y^{n}} = y^{1-n} $.

The previous equation suggest that we should do the substitution $ z = y^{1-n} $ (Also called the Bernoulli substitution), which implies:
\begin{align*}
\dfrac{dz}{dx} = (1-n) \cdot y^{-n} \dfrac{dy}{dx}
\end{align*}
or in a more suggestive way
\begin{align*}
\left( \dfrac{1}{1-n} \right)  \dfrac{dz}{dx} =   y^{-n} \dfrac{dy}{dx}
\end{align*}
so that \eqref{eq:Bernoulli Substituion} transforms into:
\begin{align*}
\left( \dfrac{1}{1-n} \right)  \dfrac{dz}{dx} + P(x)z = f(x)
\end{align*}
or 
\begin{align*}
\dfrac{dz}{dx} + \left( 1-n \right) P(x) z = \left( 1-n \right) f(x)
\end{align*}
Which is a linear equation of the form \eqref{LinearDif}, for which we can find $z$ via \eqref{eq:linearsolution}. 
Observe that work is not finished there, at the end we need to go our original variables with the relation $ z = y^{1-n} $, that is $ y = z^{1/1-n} $
\section{The Laplace Transform}
Many problems encountered in physics and engineering involves forces that are represented by discontinuous functions, observe that many of the methods described above for solving differential equations are rather complicated or awkward to use, in this section, we will define and get acquainted with a tool that will allow us to solve a complicated differential equation, by translating the problem into an algebraic problem, which is more easy to solve.
\subsection{Mathematical Background}
The following section contains a small discussion on improper integrals and the existence theorems of the Laplace transform, as such, it should be glimpsed.
\begin{define}
	An improper integral over an bounded interval is defined as a limit of integrals over finite intervals; thus
	\begin{align*}
	\int_0^\infty f(t) dt = \lim_{A \rightarrow \infty} \int_a^{A} f(t) dt
	\end{align*}
	if the limit exist, we say that the improper integral \textbf{converges}. Otherwise we say that the integral \textbf{diverges}.
\end{define}
The following definitions will be important for studying the existence of improper integrals.
\begin{define}
	A function $f$ is said to be \textbf{piecewise continuous} on an interval $ \alpha \leq t \leq \beta $. If there exist a partition $P = \left\lbrace \alpha = t_0 < t_1 < \ldots t_n = \beta \right\rbrace $ so that:
	\begin{itemize}
		\item $f$ is continuous in each open subinterval $(t_{i-1},t_i)$
		\item $f$ approaches a finite limit as the endpoints of each subinterval are approached from within the subinterval
	\end{itemize}
\end{define}
In different words, we say that $f$ is piecewise continuous on an interval, if it is continuous there except for a finite number of jump discontinuities (not easy to show). 
Now since we are studying the existence of the improper integral, observe that if the integral has a closed form solution, i.e, you can ``evaluate'' the integral, it is usually simple to see if the limit exist. Observe then, that we cannot say much of the existence of the limit, if we do not have a closed form solution. For circumventing this problem, we will make use of the following theorem that will allow us to compare and test improper integrals:
\begin{thm}
	If $f$ is piecewise continuous for $t \geq a$, if $ \vert f(t) \vert \leq g(t) $ when $t \geq M$ for some positive constant $M$, and if $ \int_M^\infty g(t)dt $ converges, then $ \int_a^{\infty} f(t) dt $ also converges. On the other hand, if $f(t) \geq g(t) \geq 0 $ for $ t \geq M$, and if $ \int_M^{\infty} g(t) dt $ diverges, then $ \int_a^{\infty} f(t) dt $ also diverges
\end{thm} 
The proof of this theorem can be found in any Calculus book, or it can be thoroughly studied in a real analysis class.
However, observe that our geometric intuition can give us an image of why the preceding theorem is true, in the way of comparing areas of the functions.
Now, we will see that the Laplace transform is an special transform of a larger set of integral transforms.
\begin{define}
	An \textbf{integral transform} is a relation of the form:
	\begin{align*}
	F(s) = \int_{\alpha}^{\beta} K(s,t)f(t) dt
	\end{align*}
	Where $K$ is a given function known as the \textbf{kernel} of the transformation. In all cases, $ \alpha $ and $ \beta$ are given, and are elements of the extended real line ($\mathbb{R} \cup \left\lbrace  -\infty,\infty \right\rbrace$) 
	\end{define}
Observe that the previous relation is acting on $f$, that is, we are giving an arbitrary function $f$ and we are getting a different function $F$.

There are many useful integral transformation in applied mathematics, but for this section we will use the Laplace transform, which is defined by:
\begin{define} \label{def:Laplace}
	The \textbf{Laplace transform} of a function $f$ is given by:
	\begin{align*}
	\LL\braces{f(t)} = F(s) = \int_0^{\infty} e^{-st}f(t)dt
	\end{align*}
\end{define}
The general idea of using the Laplace transform is:
\begin{itemize}
	\item Use the previous relation to transform a differential equation on $f$ in the $t$ domain and then transforming into an algebraic problem for the transform $F$ in the variable $s$
	\item Solve the algebraic problem to find $F$
	\item Recover the function $f$ from $F$, this last step is usually ``inverting'' the transform
\end{itemize}
The problem now is to investigate, when does the Laplace transform exist, the following theorem provides us an answer:
\begin{thm}
	Suppose that
	\begin{itemize}
		\item $f$ is piecewise continuous on the interval $ 0 \leq t \leq A$ for any positive $A$ 
		\item $ |f(t)| \leq Ke^{at} $ where $ t \geq M$. In this inequality, $K, a$, and $M$ are real constants, $K$ and $M$ necessarily positive
	\end{itemize}
	Then the Laplace transform $ \LL \braces{f(t)} = F(s)$, defined by Definition \ref{def:Laplace}, exist for $ s > a$.
\end{thm}

\begin{proof}
	We want to see that the integral in Definition \ref{def:Laplace} converges for $ s > a$. So observe that we can write the integral as:
	\begin{align}
	\int_0^\infty e^{-st}f(t)dt = \int_0^M e^{-st}f(t)dt + \int_M^\infty e^{-st}f(t)dt
	\end{align}
	and we see that the first integral of the right hand side exists with the first of the hypotheses of the theorem. and now for the second integral, we observe that the second of the hypotheses we have, for $t \geq M$,
	\begin{align*}
	|e^{-st}| \leq Ke^{-st}e^{at} = Ke^{a-s}t
	\end{align*}
	And thus by comparison (the comparison theorem in this section), we see that $F(s)$ exists provided that $ \int_M^{\infty} e^{(a-s)t} dt $ converges. We see then that this integral only converges when $ (a-s) < 0$, or equivalently $a<s$. Which establish the theorem.
\end{proof}
A few remarks are in order. First, as in the book, class, and this notes, we deal exclusively with functions that satisfy the conditions of the above theorem, although this may seem that we are reducing our library of functions, we can say that these functions will be sufficient for physics and applied mathematics (some obscure functions may appear, but they are the \emph{pathological} examples created by pure mathematicians). Second, the functions that satisfy the above theorem are called as piecewise continuous and of \textbf{exponential order} as $ t \rightarrow \infty$. (For more information on this subjects look up Big-O notation).

Now, we have just shown that the Laplace transform is an application from the set of functions that are of \textbf{exponential order} to  the set of function of s-domain. Now we want to be able to define a function from functions of s-domain to the functions of exponential order. The following theorem provides us with a tool:

\begin{thm}
	(\textbf{Informal Lerch's Theorem} - Uniqueness of Inverese Laplace Transforms). Suppose $f(t)$ and $g(t)$ are continuous on $ [, \infty)$ and of exponential order $ \gamma$. If $ \laplace{f(t)} = \laplace{g(t)} $ for all  $s > \gamma$, then $f(t) = g(t)$ for all $ t \geq 0$
\end{thm}

Strictly speaking this theorem is false in general, but for many (mostly all of them) functions in applied mathematics, the theorem above is true.
The formal version of the above theorem is:
\begin{thm}
	\textbf{Lerch's Theorem}. If there are two functions $F_1(t)$ and $F_2(t)$ with the same integral transform:
	\begin{align*}
	\laplace{F_1(t)} = \laplace{F_2(t)} \equiv f(s)
	\end{align*}
	then a \textbf{null function} can be defined by:
	\begin{align*}
	\delta_0(t) = F_1(t) - F_2(t)
	\end{align*}
	so that the integral
	\begin{align*}
	\int_0^a \delta_0(t) dt = 0
	\end{align*}
	vanishes for all $a>0$
\end{thm} 

With these theorems we can then define an \textbf{inverse Laplace transform}:

\begin{define}
	Given a function $f$ that has a Laplace transform, that is:
	\begin{align*}
	\laplace{f(t)} = F(s)
	\end{align*}
	We define the \textbf{Inverse Laplace transform} as:
	\begin{align*}
		\laplaceinv{F(s)} = f(t)
	\end{align*}
\end{define}
Although this seem a rather empty definition, we do not provide a direct method to get Laplace inverse transforms, rather we just spent doing a lot of Laplace transforms and defining the inverse Laplace transform with the above definition
\subsection{Using the Laplace Transform}
We define the Laplace Transform as:
\begin{define} \label{def:Laplace}
	The \textbf{Laplace transform} of a function $f$ is given by:
	\begin{align*}
	\LL\braces{f(t)} = F(s) = \int_0^{\infty} e^{-st}f(t)dt
	\end{align*}
\end{define}
But as you will see, it is rather cumbersome to work with the definition, let us see that with an example. Suppose that we want to know the Laplace transform of $f(t) = \sin(at) $ 
\begin{mdframed}
\textbf{Example} Compute the Laplace transform of $ f(t) = \sin(at) $ for $ t \geq 0 $.
\begin{align*}
\LL\braces{\sin(at)} = F(s) = \int_0^{\infty}e^{-st}\sin(at)dt, \quad s>0
\end{align*}
Equivalently,
\begin{align*}
F(s) = \lim_{A \rightarrow \infty} \int_0^{A}e^{-st}\sin(at)dt
\end{align*}
We integrate by parts. Using $u = e^{-st}$ and $dv= \sin(at)$, which implies $du = -se^{-st}$ and $ v = -\frac{\cos(at)}{a}$
\begin{align*}
F(s) = \lim_{A \rightarrow \infty} \left[ \left. -\frac{e^{-st}\cos(at)}{a} \right|_0^A - \frac{s}{a}\int_0^Ae^{-st}\cos(at)dt   \right] \\
=\frac{1}{a} - \frac{s}{a}\int_0^{\infty}e^{-st}\cos(at)dt
\end{align*}
A second integration on the integral, with $ u = e^{-st}$ and $ dv = \cos(at) $, which implies $ du = -se^{-st}$ and $ v = \sin(at)$, yields
\begin{align*}
F(s) = \frac{1}{a} - \frac{s^2}{a^2}\int_0^{\infty}e^{-st}\sin(at)dt \\
F(s) = \frac{1}{a} - \frac{s^2}{a^2}F(s)
\end{align*}
Finally, solving for $F(s)$
\begin{align*}
	F(s) = \frac{a}{s^2+a^2} \quad s>0
\end{align*}
\end{mdframed}
So observe that wasn't a very useful or easy work.Before we embark on our study of Laplace transform of functions, we will need the following theorem.
\begin{thm}
	The Laplace transform is a \textbf{linear operator}, that is, the Laplace transform complies with:
	\begin{align*}
	\LL\braces{c_1f_1(t) + c_2f_2(t)} = c_1\LL\braces{f_1(t)} + c_2\LL\braces{f_2(t)}
	\end{align*} 
	Where $c_1,c_2$ are real numbers (it even holds for complex numbers, which allows for neat tricks), and $f_1,f_2$ are functions whose Laplace transforms exist.
\end{thm} 
\begin{proof}
	From the definition, Suppose $c_1,c_2$ are real numbers, and $f_1,f_2$ are functions such that the Laplace transform of these functions exists. Observe
	\begin{align*}
	\LL\braces{c_1f_1(t) + c_2f_2(t)} = \int_0^\infty e^{-st} \left[ c_1f_1(t) + c_2f_2(t) \right]dt \\
	= c_1 \int_0^\infty e^{-st}f_1(t) + c_2 \int_0^\infty e^{-st}f_2(t)
	\end{align*}
	Therefore, we get
	\begin{align*}
	\LL\braces{c_1f_1(t) + c_2f_2(t)} = c_1\LL\braces{f_1(t)} + c_2\LL\braces{f_2(t)}
	\end{align*}
	Which is the desired result.
	Observe that this proof can be generalized to a arbitrary number of terms, that is:
	\begin{align*}
	\LL\braces{\sum_{i = 1}^{n} f_i(t)} = \sum_{i = 1}^{n} \LL\braces{f_i(t)}
	\end{align*}
\end{proof}
Now, since this a course on differential equation, the main objective is to solve differential equations using the Laplace transform, the bridge will be given by the following theorem. \newline
\textbf{Remark} From now on, we will assume that the functions given are of exponential order and piecewise continuous, this is made in order to simplify the theorems.
\begin{thm}
	The Laplace transform of the nth-derivative of a function $f$, that is, $ \LL\braces{f^{(n)} (t)} $ is given by:
	\begin{align*}
	\LL\braces{f^{(n)} (t)} = s^n \LL\braces{f(t)} - s^{n-1}f(0) - \ldots -sf^{(n-2)}(0) - f^{(n-1)}(0).
	\end{align*}
\end{thm} 
\textbf{Remark} Observe that for case $n=2$ (the most encountered case in this course, worth memorizing) the theorem is telling us that:
\begin{align*}
\LL\braces{f^{(2)} (t)} = s^2\LL\braces{f(t)} - sf(0) - f(0)
\end{align*}
Let us observe this in action with an example.
\begin{mdframed}
	Consider the following differential equation:
	\begin{align*}
	\frac{d^2f}{dt^2} - \frac{df}{dt} - 2f(t) = 0
	\end{align*}
	with the initial value conditions:
	\begin{align*}
	f(0) = 1, \quad \eval{\frac{df}{dt}}{t}{0} = 0
	\end{align*}
	Although, we already know a procedure to solve this particular differential equation, let us use the tool of Laplace transform. First, let us take the Laplace transform to both sides.
	\begin{align*}
	\LL\braces{\frac{d^2f}{dt^2} - \frac{df}{dt} - 2f(t) = 0} = \LL\braces{0} \\
	\laplace{\frac{d^2f}{dt^2}} - \laplace{\frac{df}{dt}} - \laplace{2f(t)} = 0\\
	\laplace{f^{(2)}(t)} - \laplace{f^{(1)}(t)} -2\laplace{f(t)}  = 0
	\end{align*}
	And using our previous theorem we have:
	\begin{align*}
		s^2\laplace{f(t)} - sf(0) - \eval{\frac{df}{dt}}{t}{0} - \left[ s\laplace{f(t)} - f(0) \right] -2\laplace{f(t)} = 0 \\
		s^2\laplace{f(t)} - sf(0) - f^{(1)}(0) - \left[ s\laplace{f(t)} - f(0) \right] -2\laplace{f(t)} = 0 \\
	\end{align*}
	Now using the initial value conditions and letting $ \laplace{f(t)} = F(s) $ we have
	\begin{align*}
	\left( s^2 - s -2\right)F(s) + (1-s)f(0) - f^{(1)}(0) = 0
	\end{align*}
	equivalently,
	\begin{align*}
	F(s) = \frac{s-1}{s^2 - s -2} = \frac{s-1}{(s-2)(s+1)}
	\end{align*}
	Let's use partial fractions on the last inequality to get:
	\begin{align*}
	F(s) = \frac{s-1}{(s-2)(s+1)} = \frac{1/3}{s-2} + \frac{2/3}{s+1}
	\end{align*}
	Now observe that we evaluated that: $ \laplace{\frac{1}{3}e^{2t}} = \frac{1/3}{s-2} $. Similarly, $ \laplace{\frac{2}{3}e^{-t}} = \frac{2/3}{s+1}$. so that:
	\begin{align*}
	\laplace{\frac{1}{3}e^{2t}} + \laplace{\frac{2}{3}e^{-t}} = F(s) \\
	\laplace{ \frac{1}{3}e^{2t} + \frac{2}{3}e^{-t}} = F(s) = \laplace{f(t)}
	\end{align*}
	Which implies then
	\begin{align*}
	f(t) = \frac{1}{3}e^{2t} + \frac{2}{3}e^{-t}
	\end{align*}
\end{mdframed}
So observe that, we have just transformed a differential equation problem into an algebraic one, and we just work out the ``reverse'' Laplace transform at the end. This is the power of the Laplace transform.
This last part allows us to define and introduce then the \textbf{inverse Laplace transform}.
\begin{define}
	Given a function $f$ that has a Laplace transform, that is:
	\begin{align*}
	\laplace{f(t)} = F(s)
	\end{align*}
	We define the \textbf{Inverse Laplace transform} as:
	\begin{align*}
	\laplaceinv{F(s)} = f(t)
	\end{align*}
\end{define}
The existence and \emph{well definedness} of the inverse is discussed above. Although this seem a rather empty definition, we do not provide a direct method to get Laplace inverse transforms, rather we just spent doing a lot of Laplace transforms and defining the inverse Laplace transform with the above definition.
The following provides a table with the most encountered Laplace transform and (Laplace transform inverses).







\end{document}
